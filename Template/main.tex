%
%                         LaTesiTemplate             
%                               
%  Autore del codice sorgente: Daniele Rosati.
%
%  Il codice sorgente è rilasciato secondo la licenza open source MIT. 
%
%  Si rimanda al LICENSE per il testo completo della licenza, ma in parole povere siete liberi di modificare o copiare o modificare il codice in qualunque sua parte o forma.
%
%  In caso riscontriate bug, avete consigli o modifiche sul file chiedo eventualmente di farmelo sapere tramite il git della repository del progetto:
%
%  Per approfondimenti circa le opzioni ed il funzionamento del template, si rimanda alla wiki del progetto.
%
%   PREAMBOLO E OPZIONI GENERALI

\documentclass[11pt,a4paper,oneside,openright,regno]{book}

\usepackage{geometry}
\geometry{a4paper,top=3.5cm,bottom=3cm,left=3.5cm,right=3.5cm,heightrounded,bindingoffset=5mm}

\usepackage[italian]{babel}
\usepackage[T1]{fontenc}
\usepackage[utf8]{inputenc}
\usepackage{fancyhdr}
\usepackage{indentfirst}
\usepackage{graphicx}
\usepackage{mathtools}
\usepackage{setspace}
\usepackage{frontespizio}

\usepackage[bottom]{footmisc}
\usepackage{bbm}
\usepackage{booktabs}

%  Pacchetti Header e Footer delle pagine

\pagestyle{fancy}
\fancyhf{}
\fancyhead[R]{\leftmark} 
\fancyfoot[C]{\thepage}

%  Pachetto Appendici

\usepackage[toc,page]{appendix}

%  Pacchetti Grafici

\usepackage{graphicx}
\graphicspath{{images/}}

%  Pacchetti Tabelle

\usepackage{rotating} 
\usepackage{caption} 
\usepackage{adjustbox}
\usepackage[section]{placeins}
\usepackage{subcaption}

%  Pacchetto Codice Tesi

\usepackage{minted}

%  Pacchetti Bibliografia

\usepackage[backend=biber, style=bwl-FU]{biblatex}
\addbibresource{Bibliography.bib}
\usepackage{csquotes}

%  Codice Frontespizio
%
%  Il frontespizio è differente da università ad università perciò qui ne è stato inserito uno generico.
%  Per modificarlo avete 3 strade:
%  a) la vostra università mette a disposizione il codice latex con i file necessari
%  b) utilizzare il doc con il titolo e integrarlo con il codice 
%  c) provarlo a generare voi il codice

\doublespace

\title{Titolo Tesi - Draft}
\author{Autore Titolo}
\date{Inserire Data}

%  DOCUMENTO PRINCIPALE

\begin{document}

\frontmatter
\maketitle

%  Codice per lasciare pagina in bianco; se non si vuole cancellare righe 94-99 tipo presenti nel codice.

\clearpage
\thispagestyle{empty}
\phantom{a}
\vfill
\newpage
\vfill

%  Dedica Tesi

\clearpage
\thispagestyle{empty}
\null\vspace{\stretch{1}}
\begin{flushright}
\textit{Inserire la dedica qui,\\
ricordando che con,\\
si va a capo.}
\end{flushright}
\vspace{\stretch{2}}\null

\clearpage
\thispagestyle{empty}
\phantom{a}
\vfill
\newpage
\vfill
\addtocounter{page}{-1}

%  Indice Analitico

\tableofcontents
\mainmatter
\chapter*{Introduzione}
\markboth{INTRODUZIONE}{INTRODUZIONE}
\addcontentsline{toc}{chapter}{Introduzione}
Questo file pdf è il "compilato" del codice sorgente $LaTeX$ del file Example. Nel prossimo capitolo saranno presenti alcuni stracci del testo della tesi che ho riscritto con $Latex$, con comandi che possono essere eventualmente utili per capire velocemente i comandi od il codice da utilizzare; non è necessario per capire il significato ma solo per vedere il codice come si comporta e quale usare per determinati scopi.

Questa non vuole essere assolutamente una guida o un how-to ma solo un breve testo che può essere utilizzato da coloro che cercano una breve risposta.

Online potete trovare numerosi fonti da cui cercare. Per imparare ad utilizzare $Latex$ ed eventuali dubbi, consiglio le seguenti fonti:

\begin{itemize}
\item Tex Stackexchange
\item Reddit/Latex
\item Sharelatex
\item Overleaf
\end{itemize}

Nonchè Google, of course.

Buona Fortuna e buona tesi.
\chapter{Capitolo1}
% Questo testo è un place holder, eliminalo tutto il contenuto di Introduction per iniziare a scrivere!
\chapter{Capitolo2}
% Questo testo è un place holder, eliminalo tutto il contenuto di Introduction per iniziare a scrivere!

\clearpage
\thispagestyle{empty}
\phantom{a}
\vfill
\newpage
\vfill
\addtocounter{page}{-1}

% Per aggiungere l'appendice e codice tesi è sufficiente eliminare %% nelle righe 142-148 e creare file Tex Approfondimenti e Appendici.tex e Codice Tesi.tex in chapters.
%% \appendix
%% \chapter{Approfondimenti ed Appendici}
%% \input{chapters/Appendix}
%% \renewcommand{\thesubsection}{\Alph{subsection}}
%% \chapter{Codice Tesi}
%% \input{chapters/code}

\backmatter

\clearpage
\thispagestyle{empty}
\phantom{a}
\vfill
\newpage
\vfill

\addcontentsline{toc}{chapter}{Elenco delle figure}
\listoffigures\newpage

\addcontentsline{toc}{chapter}{Elenco delle tabelle}
\listoftables\newpage

% Per aggiungere biblioografia è sufficiente eliminare %% dalle righe 166-168 e creare file Bibliography.bib 
%% \addcontentsline{toc}{chapter}{Bibliografia}
%% \nocite{*}
%% \printbibliography

\end{document}
