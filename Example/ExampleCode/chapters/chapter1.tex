In questo capitolo saranno presenti alcune codici particolari e difficoltà che ho incontrato durante lo svolgimento della tesi. Spero che quindi possa essere utile a qualcuno.

\section{Formule allineate correttamente}

Uno dei motivi per utilizzare $Latex$ è ceratamente per la capacità di gestire le formule matematiche. Uno dei problemi per i neofiti è trovare come allineare correttamente le formule.

Si rimanda al codice sorgente del pdf per capire come fare. 

\begin{align}
\frac{\sigma^2_{K_{t,T}}T}{2}&=\int_{0}^{F}\frac{1}{K^2}e^{rT}P(K)dK + \int_{F}^{\infty}\frac{1}{K^2}e^{rT}C(K)dK \nonumber \\ &=\int_{0}^{K_0}\frac{1}{K^2}e^{rT}P(K)dK + \int_{K_0}^{\infty}\frac{1}{K^2}e^{rT}C(K)dK \nonumber \\ &+ \int_{K_0}^{F}\frac{dK}{K^2}(P(K)-C(K))
\end{align}

\bigskip

Un altro problema invece è generare formule distribuite in parallelo. Può essere ostico però la logica è la stessa rispetto a quella adottata in precedenza. Si rimanda ancora una volta al codice sorgente.

\begin{align}
V_{call}=\frac{\partial c}{\partial \sigma}=\frac{S \sqrt{\tau}}{2\sigma\sqrt{2\pi}}e^{-\frac{1}{2}d^2_1} &&  V_{put}=\frac{\partial p}{\partial \sigma}=\frac{S \sqrt{\tau}}{2\sigma\sqrt{2\pi}}e^{-\frac{1}{2}d^2_1} \nonumber \\ \kappa_{call}=\frac{\partial c}{\partial \sigma^2}=\frac{S\sqrt{\tau}}{2\sigma\sqrt{2\pi}}e^{-\frac{1}{2}d_1^2} && \kappa_{put}=\frac{\partial p}{\partial \sigma^2}=\frac{S\sqrt{\tau}}{2\sigma\sqrt{2\pi}}e^{-\frac{1}{2}d_1^2}
\end{align}

\bigskip

\section{Attenzione alle tabelle!}

Una parte molto ostica e ceratamene molto rompiscatole per un neofita di Latex sono sicuramente le tabelle, in particolar modo quando queste sono molto estese e superano i margini fissati nel main. Nelle prossime pagine verranno illustrate alcune soluzioni. Si rimanda sempre al codice sorgente per vedere e capire come funzionano.

\clearpage
\thispagestyle{plain}

%%%%%%%%%%%%%%%%%%%%%%%%%%%%%%%%%%%%%%%%%%%%%%%%%%%%%%%%%%%%%%%%

\begin{sidewaystable}[htpb]
\small 
\doublespacing
\vspace{-0.0cm}
\scalebox{1.0}{
\hspace{1.0cm}
\begin{tabular}[t]{lcccccccccc}
\toprule
\bf Commodity & \bf Media & \bf Mediana & \bf Dev. Standard & \bf Max & \bf Min & \bf1 Quant & \bf 3 Quant & \bf Assimetria & \bf Curtosi & \bf AR(1) \\
\midrule
Argento & -0.0161 & -0.0145 & 0.0242 & 0.0516 & -0.2355 & -0.0279 & -0.001 & -1.0674 & 5.6796 & 0.9160\\
Mais & -0.0084 & -0.0061 & 0.0682 & 0.3382 & -0.1219 & -0.0244 & 0.0072 & 3.5634 & 13.9846 & 0.9730 \\
Oro & -0.0035 & -0.0032 & 0.0099 & 0.0391 & -0.0548 & -0.0079 & 0.0008 & 0.2097 & 6.3386 & 0.9170\\
Greggio & -0.0101 & -0.0127 & 0.0582 & 0.2808 & -0.2229 & -0.0323 & 0.0014 & 1.5819 & 7.0185 & 0.9680 \\
Soia & 0.0018 & -0.0042 & 0.0411 & 0.2077 & -0.1842 & -0.0149 & 0.0061 & 1.2727 & 6.1803 & 0.9770\\
Cacao & -0.0190 & -0.0092 & 0.0335 & 0.1161 & -0.2029 & -0.0363 & 0.0025 & -1.1233 & 5.4155 & 0.8580 \\
Bestiame & -0.0018 & -0.0028 & 0.0245 & 0.0753 & -0.1052 & -0.0157 & 0.01508 & 0.1221 & 0.6641 & 0.8412 \\
\bottomrule \\[2.0cm]
\toprule
\bf Commodity & \bf Media & \bf Mediana & \bf Dev. Standard & \bf Max & \bf Min & \bf1 Quant & \bf 3 Quant & \bf Assimetria & \bf Curtosi & \bf AR(1) \\
\midrule
Argento & -0.2453 & -0.2591 & 0.2980 & 0.4554 & -1.6882 & -0.4378 & -0.0218 & -0.4747 & 4.3576 & 0.9560\\
Mais & -0.0976 & -0.1321 & 0.5170 & 2.1829 & -1.4585 & -0.4129 & 0.1538 & 1.2002 & 3.1122 & 0.9560\\
Oro & -0.1785 & -0.1509 & 0.3040 & 0.6391 & -1.2164 & -0.3458 & 0.0242 & -0.5371 & 3.4912 & 0.9430 \\
Greggio & -0.1774 & -0.2088 & 0.3215 & 1.0067 & -1.8211 & -0.4683 & 0.0350 & 0.5968 & 3.3808 & 0.9640 \\
Soia & -0.0796 & -0.0191 & 0.5345 & 1.6427 & -1.3395 & -0.3362 & 0.3349 & 0.4037 & 1.1032 & 0.9670\\
Cacao & -0,2357 & -0,2284 & 0,5009 & 2,2749 & -2,4024 & -0,5524 & 0,1118 & -0,0512 & 4,3339 & 0,8640 \\
Bestiame & -0.0041 & -0.0402 & 0.3833 & 1.1753 & -0.2882 & 0.2210 & -0.1090  & -0.1343 & -0.1090 & 0.9413 \\
\bottomrule
\end{tabular}}
\caption{Statistiche principali associate rispettivamente a Variance Risk Premium $VRP_{t,T}$ e ad Log Variance Risk Premium $LVRP_{t,T}$ sulle commodities studiate}
\end{sidewaystable}

%%%%%%%%%%%%%%%%%%%%%%%%%%%%%%%%%%%%%%%%%%%%%%%%%%%%%%%%%%%%%%%%

\clearpage
\thispagestyle{plain}

\begin{table}[h!]
\vspace{0.8cm}
\centering \doublespacing
\small
\caption{Risultati regressione di $LVRP_{t,T}=\alpha_i + \beta_i(r^m_{t,T}-r^f_{t,T}) + \varepsilon_t$}
\scalebox{1.0}{
\begin{tabular}[h!]{lccccc}
\toprule
\bf Commodity & \bf $\alpha$ & & $\beta_i$ & & \bf $R^2$ \\
\midrule
Argento & -0.1756 & [-4.6439] & 0.0001 & [2.9184] & 0.0539\\
& & (3.664e-06) & & (2.15e-03) & \\
Mais & 0.1257 & [1.9092] & -0.0006 & [-2.1387] & 0.0958\\
& & (0.05623) & & (0.03246) & \\
Oro & -0.4813 & [-13.2807] & 0.0012 & [8.5543] & 0.2183\\ 
& & (2.2e-16) & & (2.2e-16) & \\
Greggio & 0.6187 & [13.251] & -0.0271 & [-18.160] & 0.1742\\
& & (2.2e-16) & & (2.2e-16) & \\
Soia & -0.1574 & [-2.4213] & 0.0007 &  [2.8320] & 0.0737\\
& & (0.015466) & & (0.0046) & \\
Cacao & -0.4829 & [-7.9425] & 0.0010 & [4.1626] & 0.1081\\
& & (1.981e-15) & & (3.147e-05)\\
Bestiame & -0.4829 & [-7.9425] & 0.0010 & [4.1626] & 0.0998 \\
& & (1.981e-15) & & (3.147e-05) & \\
\bottomrule
\end{tabular}}
\end{table}

%%%%%%%%%%%%%%%%%%%%%%%%%%%%%%%%%%%%%%%%%%%%%%%%%%%%%%

\clearpage
\thispagestyle{plain}

\begin{sidewaystable}[htpb]
\small 
\doublespacing
\vspace{0.0cm}
\scalebox{0.9}{
\hspace{3.0cm}
\begin{tabular}[t]{lccccccccccccccccccc}
\toprule
\bf Commodity & \bf $\alpha$ & \bf $F_{1t}$ & \bf $F_{2t}$ & \bf $F_{3t}$ & \bf $F_{4t}$ & \bf $F_{5t}$ & \bf $F_{6t}$ & \bf $F_{7t}$ & \bf $F_{8t}$ & \bf $R^2$ \\
\midrule
Argento & -0.3134 & 0.4219 & -0.2706 & -0.3901 & 0.8093 & -0.4548 & 0.2983 & 0.1016 & 0.7671 & 0.3682 \\
 & [-1.8370] & [1.2976] & [-0.3608] & [-1.1627] & [1.9108] & [-0.9561] & [0.7384] & [0.3581] & [1.8834] \\
 & (0.0662) & (0.1944) & (0.7182) & (0.2449) & (0.0560) & (0.3390) & (0.4602) & (0.7202) & (0.0596) & \\
\\
Mais & -0.4186 & 0.5770 & -1.9558 & -0.9738 & 0.6732 & -1.5183 & 0.6222 & -0.5645 & 1.0425 & 0.3476 \\
& [-1.3889] & [11.0047] & [-1.4759] & [-1.6429] & [0.8998] & [-1.8066] & [0.8719] & [-1.1258] & [1.4491] & \\
& (0.1648) & (0.0015) & (0.1399) & (0.1004) & (0.3682) & (0.0708) & (0.3832) & (0.2602) & (0.1473) & \\
\\
Oro & -0.3749 & 0.6075 & -0.2652 & -0.4038 & 0.5786 & 0.0310 & 0.2074 & 0.2407 & 0.9502 & 0.3832 \\
& [-2.5529] & [2.1708] & [-0.4102] & [-1.3979] & [1.5870] & [0.0758] & [0.5965] & [0.9851] & [2.7102] & \\
& (0.0106) & (0.0299) & (0.6812) & (0.1621) & (0.1125) & (0.9396) & (0.5508) & (0.3245) & (0.0067) & \\
\\
Greggio & 0.1233 &  0.3334 & -0.0998 & -0.4505 & -1.6409 & -1.2767 & -0.2246 & -0.3300 & -0.3771 & 0.4213 \\
&  [0.6341] & [-2.8998] & [-0.1168] & [-1.1782] & [-3.3994] & [-2.3548] & [-0.4878] & [-1.0202] & [-0.8125] &\\
& (0.5260) & (0.0018) & (0.9070) & (0.2387) & (0.0006) & (0.0185) & (0.6256) & (0.3076) & (0.4165) & \\
\\
Soia & -0.4086 & -0.4372 & -1.8796 & -0.7363 & -0.1777 & -1.1029 & 1.1976 & 0.1438 & -0.5420 & 0.3142\\
& [-2.3393] & [-2.7520] & [-1.4011] & [-1.2271] & [-0.2345] & [-1.2964] & [1.6576] & [0.2834] & [-0.7442] & \\
& (0.0185) & (0.0092) & (0.1612) & (0.2198) & (0.8146) & (0.1949) & (0.0974) & (0.7768) & (0.4568) & \\
\\
Cacao & -0.3677 & -0.7151 & 0.0134 & -0.1688 & 0.3153 & -0.0562 & 0.6580 & 0.4322 & 0.2685 & 0.1057\\
& [-1.5217] & [-1.5531] & [0.0126] & [-0.3553] & [0.5258] & [-0.0835] & [1.1501] &[1.0751] & [0.4656] &\\
& (0.1281) & (0.1204) & (0.9899) & (0.7224) & (0.5991) & (0.9334) & (0.2501) & (0.2823) & (0.6415) & \\
\\
Bestiame & 0.4385 & -0.7589 & 0.9052 & 0.3971 & -0.7304 & -1.5789 & -0.1670 & -0.5038 & -0.4001 & 0.3892 \\
& [2.5707] & [-2.0050] & [1.2031] & [1.1739] & [-1.6191] & [-2.9599] & [-0.4116] & [-1.6324] & [-0.9847] \\
& (0.0101) & (0.0449) & (0.2289) & (0.2404) & (0.1054) & (0.0030) & (0.6806) & (0.1025) & (0.3247) & \\

\bottomrule
\end{tabular}}
\caption{Risultati regressione modello di Ludvingson e Ng}
\end{sidewaystable}

\section{Come si caricano le immagini?}

Ultima cosa che può risultare pesante e rognosa è certamente caricare le immagini. Per motivi di spazio su GitHub non posso caricare le immagini, per questo vedrete dei rettangoli vuoti, ma nel codice sorgente è presente tutto il codice necessario: sarà sufficiente solamente definire l'indirizzo di dove si trovano le immagini. 

Per le immagini consiglio di creare una cartella separata chiamata Images dove, seguendo la struttura di chapters, caricate le immagini che poi latex ripesca per creare il compilato.

\bigskip
\begin{figure}[h!]
    \centering
    \caption{Payoff tra Variance Swap e Volatility Swap}
    \includegraphics[width=10cm,height=6cm]{images/chapter1/Vol-2.jpg}
    \caption*{\footnotesize{Fonte: \textit{Derman, E., M.B. Miller, (2016), The Volatility Smile, Wiley, pag. 63.}}}
\end{figure}
\bigskip

\clearpage

\begin{figure}[t!]
\begin{subfigure}{0.49\textwidth}
\includegraphics[width=\linewidth]{images/chapter2/CorrelationFunction/Silver/acfVRPSil.png}
\caption{ACF $VRP_{t,T}$ - Argento} 
\end{subfigure}\hspace*{\fill}
\begin{subfigure}{0.49\textwidth}
\includegraphics[width=\linewidth]{images/chapter2/CorrelationFunction/Silver/pacfVRPSil.png}
\caption{PACF $VRP$ - Argento} 
\end{subfigure}

\vspace{0.35cm}

\begin{subfigure}{0.49\textwidth}
\includegraphics[width=\linewidth]{images/chapter2/CorrelationFunction/Silver/acfLVRPSil.png}
\caption{ACF $LVRP_{t,T}$ - Argento} 
\end{subfigure}\hspace*{\fill}
\begin{subfigure}{0.49\textwidth}
\includegraphics[width=\linewidth]{images/chapter2/CorrelationFunction/Silver/pacfLVRPSil.png}
\caption{PACF $LVRP_{t,T}$ - Argento} 
\end{subfigure}

\vspace{0.35cm}

\begin{subfigure}{0.49\textwidth}
\includegraphics[width=\linewidth]{images/chapter2/CorrelationFunction/Corn/acfVRPCor.png}
\caption{ACF $VRP_{t,T}$ - Mais} 
\end{subfigure}\hspace*{\fill}
\begin{subfigure}{0.49\textwidth}
\includegraphics[width=\linewidth]{images/chapter2/CorrelationFunction/Corn/pacfVRPCor.png}
\caption{PACF $VRP_{t,T}$ - Mais} 
\end{subfigure}

\vspace{0.35cm}

\caption{In ordine, rispettivamente da sinistra a destra, ACF e PACF di $VRP_{t,T}$ e $LVRP_{t,T}$ su Oro e Greggio} 
\end{figure}